\section{Introduzione}

\subsection{Abstract}
Il sito "Concessionaria GREG" vuole offrire la possibilità a tutti i visitatori di ricevere informazioni riguardo i servizi offerti, le vetture disponibili, suddivisi in veicoli a noleggio e veicoli in vendita, i contatti e l'ubicazione.\\
Il sito permette di inserire, modificare ed eliminare i prodotti: queste operazioni sono permesse solamente ad un utente privilegiato, mentre tutti gli altri utenti saranno visitatori normali a cui viene garantita la visualizzazione delle pagine del sito consentite. Viene ricercata l'accessibilità in modo che chiunque possa navigare nel sito serenamente; una volta garantita tale, si vuole focalizzare l'attenzione sull'usabilità, rispettando gli standard W3C e la separazione tra struttura, presentazione e comportamento.\\
L'obiettivo finale del progetto è garantire una navigazione fluida degli utenti all'interno del sito, in modo da evitare il disorientamento e, nel caso dovesse succedere, garantire supporto per tornare all'interno del sito o dell'area cercata.\\
\\
\textbf{Il nome "Concessionaria GREG"}: in fase di progettazione è stato scelto il nome "Concessionaria GREG" in modo arbitrario, senza riferimento a persone o luoghi fisici, pertanto è da considerarsi un semplice nome.\\

\subsection{Analisi d'utenza}
Le informazioni sono fornite dal sito tramite l'utilizzo di un linguaggio informale e comprensibile a tutti così da poter essere fruibile ad una vasta cerchia di pubblico. Vengono forniti tutti i contatti utili per comunicare con i gestori del sito in caso di necessità. Concessionaria GREG si rivolge in particolar modo ad utenti interessati all'acquisto o al noleggio di vetture.\\

\pagebreak
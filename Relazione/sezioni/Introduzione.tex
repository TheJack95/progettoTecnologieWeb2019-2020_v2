\section{Introduzione}

\subsection{Abstract}
Il progetto Concessionaria GREG vuole implementare un sito Internet che offra la possibilità di fornire informazioni riguardo i prodotti disponibili, suddivisi in veicoli a noleggio e veicoli in vendita, i contatti, l'ubicazione. Il sito permette di inserire, modificare ed eliminare i prodotti: queste operazioni sono permesse solamente ad un utente privilegiato, mentre tutti gli altri utenti saranno visitatori normali a cui viene garantita la visualizzazione delle pagine del sito consentite. Inoltre deve essere garantita l'accessibilità in modo che chiunque possa navigare nel sito serenamente. Una volta garantita l'accessibilità, si vuole focalizzare l'attenzione sull'usabilità, rispettando gli standard W3C e la separazione tra struttura, presentazione e comportamento. L'obiettivo del sito Internet è garantire una navigazione e ricerca fluida degli utenti all'interno del sito, in modo da evitare il disorientamento e, nel caso dovesse succedere, garantire supporto per tornare all'interno del sito.

\subsection{Analisi d'utenza}
Le informazioni sono fornite dal sito tramite l'utilizzo di un linguaggio informale e comprensibile così da poter essere fruibile una vasta cerchia di pubblico. Vengono forniti tutti i contatti utili per comunicare con lo staff in caso di necessità. Concessionaria GREG si rivolge in particolar modo ad utenti interessati all'acquisto o al noleggio di una vettura.

\pagebreak
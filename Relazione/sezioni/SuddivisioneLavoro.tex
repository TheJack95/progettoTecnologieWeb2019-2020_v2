\section{Suddivisione del lavoro}
Di seguito si elencano le parti realizzate da ciascun componente del gruppo:
\begin{itemize}
    \item Conte Riccardo:
        \begin{enumerate}
        \item pagine HTML: ;
        \item CSS pagine HTML ;
        \item script PHP: ;
        \item pagine PHP: ;
        \end{enumerate}
    \item Corrò Giacomo:
        \begin{enumerate}
            \item pagine HTML: noleggioAcquista, noleggioVeicolo, pagina404, paginaVuota;
            \item CSS pagine HTML noleggioAcquista, noleggioVeicolo e paginaVuota, parte del CSS del menù, CSS stampa pagine relative ai veicoli;
            \item script PHP: acquistaVeicoloFn, noleggiaVeicoloFn, funzioniVeicoli, parte di funzioniGenerali;
            \item pagine PHP: acquistaVeicoli, noleggioVeicoli, noleggioVeicolo, paginaVuota, pagina404;
            \item funzioni JS veicoli.js;
        \end{enumerate}
    \item Fiorese Giulia:
        \begin{enumerate}
            \item pagine HTML: ;
            \item CSS pagine HTML;
            \item script PHP: ;
            \item pagine PHP: ;
        \end{enumerate}
    \item Tabacchi Erik:
        \begin{enumerate}
            \item pagine HTML: ;
            \item CSS pagine HTML;
            \item script PHP: ;
            \item pagine PHP: ;
            \item database.
        \end{enumerate}
\end{itemize}
Il punto di partenza dello sviluppo è stato la creazione le pagine in HTML, definendo prima l'header, il footer e il menù insieme e poi inserendo i contenuti separatamente. Successivamente le pagine HTML sono state convertite in pagine PHP a cui è stato aggiunto il contenuto vero e proprio, ovvero quello proveniente dal database, il quale è stato sviluppato in parallelo alle pagine PHP. \\
Il passo successivo è stato scrivere tutte quelle funzioni PHP per l'interazione con il database, contenute nel file sqlInteractions.php. Per non appesantirlo troppo, quegli script che richiedevano maggiori controlli sono stati inseriti in file separati (esempio salvaAnimale.php); ciò ha reso l'attività di debug molto più veloce e mirata ma ha anche permesso di gestire al meglio eventuali errori di inserimento dei dati da parte dell'utente, fornendo messaggi d'errore precisi.
In seguito, una parte dei controlli sulla validità degli input dei form, sono stati tradotti in JavaScript, in modo da effettuare un primo controllo client side.\\
L'ultima parte della realizzazione è stata aggiungere CSS al sito, avendo cura che la visualizzazione del sito fosse la stessa su vari dispositivi desktop. Per quando riguarda l'aspetto mobile, abbiamo apportato modifiche di layout in modo da rendere il sito accessibile anche da smartphone.\\
Per quanto riguarda la relazione, ognuno ha scritto la parte di analisi relativa alla propria sezione del sito. Le parti generali della relazione sono state scritte da tutti i membri del gruppo.
Vengono elencate nel dettaglio le varie parti:
\begin{itemize}
    \item Riccardo Conte: 
    \item Giacomo Corrò: 
    \item Giulia Fiorese: 
    \item Erik Tabacchi:
\end{itemize}
Tutte le parti della relazione non elencate sono state scritte durante gli incontri, quindi tutto il gruppo ha preso parte alla stesura.
\pagebreak

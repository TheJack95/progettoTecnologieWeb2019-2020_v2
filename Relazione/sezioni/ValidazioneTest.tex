\section{Validazione e Fase di Test}

\subsection{Strumenti per la validazione}
Tutte le pagine e il relativo codice sono stati sottoposti a una validazione attraverso i seguenti strumenti:
\begin{itemize}
	\item Validatore online W3C per il CSS \url{https://jigsaw.w3.org/css-validator/};
	\item Validatore online W3C per l'HTML \url{https://validator.w3.org/};
	\item Validatore Total Validator Test per l'HTML;
	\item Contrast Analyzer per verificare che i colori usati \url{https://webaim.org/resources/contrastchecker/}.
\end{itemize}
La validazione del codice HTML è stata fatta incollando direttamente il codice sorgente prodotto dalle pagine PHP nel validatore W3C, in modo tale da poter validare tutto quel codice non presente nelle pagine in HTML.

\subsection{Strumenti per il testing}
Il sito è stato inoltre testato sui seguenti browser:
\begin{itemize}
	\item Safari;
	\item Microsoft Internet Explorer/Edge;
	\item Microsoft Edge (beta) basato su Chromium;
	\item Google Chrome;
	\item Mozilla Firefox.
\end{itemize}

\subsection{Test per l'accessibilità}

\subsection{Test manuali per il mobile}
Il sito è stato testato su dispositivi Android e iOS e presenta layout o comportamenti differenti (ad esempio le select) per cause che dipendono dal sistema operativo. Abbiamo cercato di renderlo il più utilizzabile possibile, tuttavia l'area amministratore presenta un layout che non si adattava alla visualizzazione su schermi di piccole dimensioni come quelle degli smartphone.

\subsection{Eccezioni}

\pagebreak
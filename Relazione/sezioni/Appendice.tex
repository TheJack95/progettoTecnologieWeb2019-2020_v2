\appendix
    \section{Note}
    Per scelta del gruppo ogni utente che si registra al sito è considerato generico, ovvero non amministratore. Per inserire un nuovo amministratore bisogna prima registrarsi attraverso l'apposito form e poi cambiare attraverso un query SQL o phpMyAdmin il campo FlAdmin della tabella Utenti da 0 a 1.

    \section{Popolamento}
    Per avere un reale utilizzo del sito, sono già stati inseriti alcune auto. Inoltre per l'utente generico sono stati effettuati degli acquisti e aggiunti dei messaggi visualizzabili nell'area privata.

    \section{Strumenti utilizzati}
    \begin{itemize}
        \item \textbf{GitHub:} per il versionamento; 
        \item \textbf{Visual Studio Code:} editor per la scrittura del codice; 
        \item \textbf{XAMPP:} per testare sopratutto la parte dinamica del sito (PHP e SQL);
        \item \textbf{Draw.io:} usato per UML e bozza grafica del sito;
        \item \textbf{Telegram:} per le comunicazioni interne al gruppo;
        \item \textbf{\LaTeX:} per la stesura della relazione.
    \end{itemize}
\pagebreak
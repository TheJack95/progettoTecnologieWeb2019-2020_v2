\section{Comportamento}
Il comportamento del sito, ovvero i meccanismi di risposta all'interazione con l'utente, è suddiviso in script JavaScript (chiamati a lato client) e funzioni PHP (gestite a lato server).
    \subsection{JavaScript}
        Il linguaggio JavaScript è stato utilizzato principalmente per la validazione dei dati inseriti in input dall'utente lato client e per la visualizzazione del menù con dispositivi mobile o schermi di piccole dimensioni; inoltre viene impiegato per la gestione client-side dei noleggi.
        I controlli JavaScript vengono eseguiti anche tramite PHP a lato server, in modo tale da mantenere la completa accessibilità del sito anche nel caso in cui JavaScript non fosse disponibile perché disabilitato o non supportato dal browser.
    \subsection{PHP}
        La componente PHP risulta sicuramente come la più corposa del sito in quanto viene utilizzata non solo per i controlli, ma anche per la generazione della struttura delle pagine e le interazioni con il database.
        Generare la struttura tramite PHP permette numerosi vantaggi, come:
        \begin{itemize}
            \item Il contenuto del sito rispecchia sempre il Database grazie alle estrazioni presenti nelle pagine PHP.
            \item Il sito è più mantenibile poiché le parti comuni alle pagine del sito sono scritte tramite una sola funzione PHP.
            \item Tramite la sostituzione di stringhe è possibile cambiare il codice HTML, ad esempio per togliere i link circolari.
        \end{itemize}
\pagebreak

\section{Analisi}

    \subsection{Utenza}
        L'utenza del sito non può essere inquadrata in un target specifico poiché destinata al pubblico della Concessionaria GREG il quale non presenta restrizioni o interesse per gruppi definiti. Ciò si traduce nell'esigenza di un sito dalla facile e immediata comprensione e nella piena ottemperanza dei canoni di accessibilità.

    \subsection{Casi d'uso}

        \subsubsection{Utente generico}
        L’utente viene definito generico nel momento in cui può solo navigare all’interno del sito, senza alcun permesso di accedere all’area riservata, richiere preventivi o effettuale noleggio di automobili.
        L’utente generico dispone dei seguenti casi d’uso:
        \begin{itemize}
            \item Visualizzazione pagina "Home";
            \item Visualizzazione pagina "Veicoli a noleggio";
            \item Visualizzazione pagina "Veicoli in vendita";
            \item Visualizzazione pagina "Contatti";
            \item Ricerca veicoli nelle due pagine precedenti;
            \item Invio messaggi attraverso form nella pagina "Contatti";
            \item Visualizzazione pagina "Login";
            \item Visualizzazione pagina "Registrazione";
            \item Possibilità di registrarsi;
        \end{itemize}

        \paragraph{Visualizzazione pagina "Home"}
        La homepage del sito, la prima a cui l'utente accede, deve contenere le 5 W e presentare all'utente una descrizione della politica e degli obbiettivi della concessionaria.
        Prevede 2 sezioni:
        \begin{enumerate}
            \item Una breve presentazione della concessionaria e delle politiche aziendali; 
            \item una sezione che conterra l'occasione del giorno;
            \end{enumerate}
        Se si desidera raggiungere la homepage mentre si sta visitando una pagina interna del sito, è possibile cliccare sulla voce home del menù oppure sul titolo/logo presente nell'header.

        \paragraph{Visualizzazione pagina "Veicoli a noleggio"}
        In questa pagina vengono visualizzati i veicoli disponibili per il noleggio. L'utente può esegueguire una ricerca attraverso l'apposita sezione "Filtri di ricerca". Nel caso in cui l'utente provasse a noleggiare un veicolo, non essendo autenticato, gli verrà mostrato un messaggio d'errore.

        \paragraph{Visualizzazione pagina "Veicoli in vendita"}
        In questa pagina vengono visualizzati i veicoli disponibili per l'acquisto. L'utente può esegueguire una ricerca attraverso l'apposita sezione "Filtri di ricerca". Nel caso in cui l'utente provasse a richiedere un preventivo per un veicolo, non essendo autenticato, gli verrà mostrato un messaggio d'errore.

        \paragraph{Visualizzazione pagina "Contatti"}
        L’utente generico accede alla pagina Contatti cliccando la sezione Contatti presente nella navbar; questa è l'unica pagina che è stata dichiarata come HTML 5 nel sito dato che al suo interno si trova una mappa importata da google maps. La pagina offre informazioni e servizi utili all’utente come:
        \begin{itemize}
            \item Un indirizzo email;
            \item Un numero di telefono;
            \item Un numero di fax;
            \item Una mappa con le indicazioni per raggiungere la concessionaria importata da Google Maps;
            \item Un form che l’utente può utilizzare per mandare messaggi al personale della concessionaria.
        \end{itemize}
        Il form è composto dai campi nome, cognome, email, numero telefono e messaggio tutti da completare obbligatoriamente, dopo essere stato inoltrato il form esegue uno script php per inviare il messaggio in una sezione dedicata del database. Il form è utilizzabile anche da utenti non registrati. \\
        \textbf{Nota:} nel caso in cui un utente non registrato effettuasse la registrazione con la stessa mail con cui ha inviato dei messaggi, questi saranno presenti nella sua area privata.

        \subsubsection{Utente registrato}
        L’utente viene definito registrato nel momento in cui ha effettuato la registrazione; dispone dei degli stessi casi  d’uso dell'utente generico con l'aggiunte della possibilità di effettuare il login.

        \subsubsection{Utente autenticato}
        L’utente viene definito autenticato nel momento in cui ha effettuato l'accesso al sito con le proprie credenziali.
        L’utente autenticato dispone dei degli stessi casi casi d’uso dell'utente generico con l'aggiunte dei seguenti
        \begin{itemize}
            \item Accesso all'area presonale
            \begin{itemize}
                \item Visualizzazione pagina "Dati Personali";
                \item Modificare i dati personali;
                \item Visualizzazione pagina "Preventivi";
                \item Visualizzazione pagina "Noleggi";
                \item Visualizzazione pagina "Messaggi";
            \end{itemize}
            \item Acquisto di un veicolo;
            \item Noleggio di un veicolo;
        \end{itemize}

        \paragraph{Area personale} Un utente generico che accede alla sua area privata, come prima pagina vedrà le azioni rapide che può eseguire, come leggere i messaggi, fare acquisti o contattare l'amministratore. Attraverso il "Menu Utente" potrà navigare all'interno dell'area privata e accedere alle 4 sottosezioni presenti: la pagina dei messaggi in cui visualizzare le conversazioni con l'amministratore, lo storico dei noleggi e dei preventivi richiesti, la pagina di visualizzazione e  modifica dei dati personali.

        \paragraph{Acquisto di un veicolo}
        L'utente autenticato può acquistare un veicolo visitando la pagina "Veicoli in vendita" e cliccando l'apposito pulsante posto in corrispondenza del veicolo desiderato.

        \paragraph{Noleggio di un veicolo}
        L'utente autenticato può noleggiare un veicolo visitando la pagina "Veicoli a noleggio" e cliccando l'apposito pulsante posto in corrispondenza del veicolo desiderato. Verrà rendindirizzato in una pagina in cui dovrà inserire le date di inizio e fine del pediodo di noleggio; in caso di date incoerenti o indisponibilità del veicolo, all'utente verrà mostrato un messaggio d'errore.

        \subsubsection{Utente amministratore}
        Il sistema riconosce l’amministratore tramite il flag presente e modificabile solo dal database.
        L’accesso avviene tramite login classica con email e password.
        L’utente autenticato come amministratore dispone degli stessi casi casi d’uso dell'utente generico con l'aggiunta delle seguenti funzionalità
        \begin{itemize}
            \item Accesso all'area presonale dell'amministratore;
            \item Vedere e modificare le proprie informazioni personali;
            \item Visualizzare la lista dei veicoli a noleggio e in vendita;
            \item Visualizzare ed eliminare le prenotazioni dei veicoli a noleggio;
            \item Aggiungere, modificare ed eliminare un veicolo;
            \item Visualizzare la lista completa dei messaggi e delle eventuali risposte inviate;
            \item Rispondere ai messaggi ricevuti.
        \end{itemize}

        \paragraph{Area amministratore} L'area privata dell'utente amministratore presenta due importanti funzionalità: l'inserimento e la cancellazione di un auto. Per l'inserimento è sufficiente cliccare sui pulsanti presenti nelle "Azioni Rapide", mentre nelle sottosezioni dell'area amministrazione l'amministrazione può consultare ed eliminare i dati presenti nel database. Può inoltre consultare gli acquisti e i messaggi di tutti gli utenti registrati. \\
        \textbf{Note:} questa parte del sito, nonostante sia molto simili per funzionalità all'area personale di un utente generico, presenta dei colori diversi proprio per contraddistinguerla.  
\pagebreak
